\documentclass[a4paper,openright,twoside]{report}

%====================== PACKAGES ======================

\usepackage[french]{babel}
\usepackage[utf8x]{inputenc}
%pour gérer les positionnement d'images
\usepackage{float}
\usepackage{amsmath}
\usepackage{graphicx}
%\usepackage[colorinlistoftodos]{todonotes}
\usepackage{url}
%pour les informations sur un document compilé en PDF et les liens externes / internes
\usepackage{hyperref}
%pour la mise en page des tableaux
\usepackage{array}
\usepackage{tabularx}
%pour utiliser \floatbarrier
%\usepackage{placeins}
%\usepackage{floatrow}
%espacement entre les lignes
%\usepackage{setspace}
%%modifier la mise en page de l'abstract
%\usepackage{abstract}

%%police et mise en page (marges) du document
\usepackage[T1]{fontenc}
\usepackage[top=2.2cm, bottom=2.2cm, left=2.2cm, right=2.2cm]{geometry}
%Pour les galerie d'images
\usepackage{subfig}
\renewcommand{\baselinestretch}{1.25}   %Taille Interligne
%Font plus moderne
\usepackage{lmodern}
%espacement entre les lignes d'un tableau
\renewcommand{\arraystretch}{1.5}

%====================== INFORMATION ET REGLES ======================

%\hypersetup{							% Information sur le document
%pdfauthor = {Premier Auteur,
			%Deuxième Auteur,
			%Troisième Auteur},			% Auteurs
%pdftitle = {Nom du Projet -
			%Sujet du Projet},			% Titre du document
%pdfsubject = {Mémoire de Projet},		% Sujet
%pdfkeywords = {Tag1, Tag2, Tag3, ...},	% Mots-clefs
%pdfstartview={FitH}}					% ajuste la page à la largueur de l'écran
%%pdfcreator = {MikTeX},% Logiciel qui a crée le document
%%pdfproducer = {}} % Société avec produit le logiciel

%======================== DEBUT DU DOCUMENT ========================

\begin{document}
%%%%%%%%%%%%%%%%%%%%%%%%%%%%%%%%
%page de garde
\thispagestyle{empty}
%régler l'espacement entre les lignes
\newcommand{\HRule}{\rule{\linewidth}{0.5mm}}
\begin{titlepage}
\begin{center}

% Upper part of the page. The '~' is needed because only works if a paragraph has started.
\includegraphics[width=0.35\textwidth]{./images/image1}~\\[2cm]

%\textsc{\LARGE INSA TOULOUSE}\\[1.5cm]

\textsc{\Large }\\[2.5cm]

% Title
\HRule \\[0.4cm]

{\huge \bfseries BE - Graphes\\
3-MIC, Groupe ? \\[0.4cm] }

\HRule \\[1.5cm]

\textsc{\Large }\\[1.5cm]


% Author and supervisor
\begin{minipage}{0.4\textwidth}
\begin{flushleft} \large
\emph{Auteur:}\\
Premier \textsc{Auteur}\\
Deuxième \textsc{Auteur}\\
Troisième \textsc{Auteur}
%\\
%Quatrième \textsc{Auteur}
\end{flushleft}
\end{minipage}
\begin{minipage}{0.4\textwidth}
\begin{flushright} \large
%\emph{Client:} \\
%Prénom \textsc{Nom}\\
\emph{Encadrant:} \\
Prénom \textsc{Nom}
\end{flushright}
\end{minipage}

\vfill

\textsc{\LARGE INSA TOULOUSE}\\[1.5cm]

% Bottom of the page
{\large \today}

\end{center}
\end{titlepage}


%%%%%%%%%%%%%%%%%%%%%%%%%%%%%%%%
% Table des matières
\thispagestyle{empty}
\setcounter{page}{0}
\tableofcontents
%\newpage

%%%%%%%%%%%%%%%%%%%%%%%%%%%%%%%%
% Les différents chapitres 
%~~
%\thispagestyle{empty}
%%recommencer la numérotation des pages à "1"
%\setcounter{page}{0}

\chapter{Présentation du Bureau d'Etudes Graphes}

Attention : vérifiez que le chapitre commence sur une page recto.

\section{Le sujet}
Décrivez le sujet et donnez l'objectif de ce document 
\\
1 page maximum

\chapter{Validation des algorithmes de calcul de plus court chemin}

Introduction : expliquez (en 2 ou 3 phrases) l'objectif de ce chapitre et son organisation

\section{Tests sur le code produit}

\subsection{Contexte}
Précisez le contexte des tests effectués : 
\begin{itemize}
	\item cartes utilisées (routières, non routières, taille des cartes, ...),
	\item nombre de tests effectués sur chaque carte, 
	\item  déroulement des tests,
	\item ....
\end{itemize}

\subsection{Tests des cas d'erreur}
Traitement d'erreurs dans les données (sommets inexistants, arcs inexistants, arcs négatifs, ....)
\\
Expliquez comment vous avez procédé à ces vérifications.


\subsection{Absence de chemin ou chemins de cout nul}
Détection de l'absence de chemin entre deux sommets et de chemins de cout nul. 
\\
Expliquez comment vous avez procédé à ces vérifications.


\subsection{Cout des chemins (1)}
Le cout (en distance et en temps) retournés par les algorithmes (Dijkstra et A$^*$) est-il : 
\begin{itemize}
	\item identique ?
	\item le même que celui retourné par la fonction \texttt{calcul\_cout\_chemin} ?
\end{itemize}
Expliquez comment vous avez procédé à ces vérifications.
\newline

\subsection{Vérification du théorème sur les plus courts chemins}
Un sous-chemin d'un plus court chemin est-il un plus court chemin (en distance et en temps) ? 
\\
Expliquez comment vous avez procédé à ces vérifications.


\subsection{Cout des chemins (2) (option)}
\noindent
Comparez le cout en temps de plus courts chemins en distance par rapport aux plus courts chemins en temps.
\\
Comparez le cout en distance de plus courts chemins en distance par rapport aux plus courts chemins en distance.
\\
Expliquez comment vous avez procédé à ces vérifications.

\subsection{Inégalité triangulaire (option)}
L'inégalité triangulaire est-elle vérifiée ? 
\\
Expliquez comment vous avez procédé à ces vérifications.


\subsection{Bilan récapitulatif des tests}
Résumez les résultats obtenus.
\\
Expliquez comment reproduire vos tests à partir de votre code source

\section{Comparaisons croisées des résultats obtenus}
\subsection{Comparaisons avec d'autres binômes}
Résultats obtenus par rapport à d'autres binômes sur des couples (origine, destination) fixés : pour chaque algorithme vous devez donner le cout de vos solutions (dans les 2 cas, distance et temps), les temps de calcul, le nombre de sommets explorés (ie entrant dans le tas), et le nombre maximum de sommets dans le tas.

Dans le tableau~\ref{tab:validation-comparaison-resultats-distance}, donnez les résultats obtenus pour les plus courts chemins en distance et
dans le tableau~\ref{tab:validation-comparaison-resultats-temps}, donnez les résultats obtenus pour les plus courts chemins en temps.
%tableau centré à taille variable qui s'ajuste automatiquement suivant la longueur du contenu
\begin{figure}[!h]
\begin{center}
\begin{tabular}{|l||l|l|l|l||l|l|l|l|}
  \hline
    PCC Distance   & \multicolumn{4}{c||}{Algorithme Dijktra} & \multicolumn{4}{c|}{Algorithme A$^*$}\\
  \hline
	 Trajets & Cout & CPU & Nb Tas & Max Tas & Cout & CPU & Nb Tas & Max Tas \\
  \hline
	O1, D1 & & & & & & & & \\
	\hline
	O2, D2 & & & & & & & & \\
	\hline
	O3, D3 & & & & & & & & \\
	\hline
\end{tabular}
\end{center}
\caption{Tableau récapitulatif des solutions pour les PCC en distance}
\label{tab:validation-comparaison-resultats-distance}
\end{figure}

%tableau centré à taille variable qui s'ajuste automatiquement suivant la longueur du contenu
\begin{figure}[!h]
\begin{center}
\begin{tabular}{|l||l|l|l|l||l|l|l|l|}
  \hline
   PCC Temps & \multicolumn{4}{c||}{Algorithme Dijktra} & \multicolumn{4}{c|}{Algorithme A$^*$}\\
  \hline
	 Trajets & Cout & CPU & Nb Tas & Max Tas & Cout & CPU & Nb Tas & Max Tas \\
  \hline
	O1, D1 & & & & & & & & \\
	\hline
	O2, D2 & & & & & & & & \\
	\hline
	O3, D3 & & & & & & & & \\
	\hline
\end{tabular}
\end{center}
\caption{Tableau récapitulatif des solutions pour les PCC en temps}
\label{tab:validation-comparaison-resultats-temps}
\end{figure}

\paragraph*{Bilan de cette comparaison}

\subsection{Comparaisons avec un autre calculateur (option)}
Comparez vos résultats avec un calculateur d'itinéraires existant.

\subsection{Vérification visuelle (option)}

\subsection{Bilan récapitulatif du comparatif croisé}
Résumez les résultats obtenus.
\\
Expliquez comment reproduire vos tests à partir de votre code source

\section{Conclusion}

\chapter{Analyse des performances}

Introduction : expliquez (en 2 ou 3 phrases) l'objectif de ce chapitre et son organisation

\section{Contexte expérimental}
Précisez le contexte des tests effectués : 
\begin{itemize}
	\item cartes utilisées (routières, non routières, taille des cartes, ...),
	\item nombre de tests effectués sur chaque carte, 
	\item  déroulement des tests,
	\item ....
\end{itemize}
Expliquez comment reproduire vos tests de performance à partir de votre code source

\section{Caractéristiques de l'algorithme Dijkstra}
Evaluez les caractéristiques de l'algorithme en cout des solutions (distance et temps), temps de calcul, nombre de sommets explorés (ie entrant dans le tas), et nombre maximum de sommets dans le tas (vous pouvez considérer des caractéristiques supplémentaires).

\subsection{Cartes routières}

\subsection{Cartes non routières}

\section{Caractéristiques de l'algorithme A$^*$}
Evaluez les caractéristiques de l'algorithme en cout des solutions (distance et temps), temps de calcul, nombre de sommets explorés (ie entrant dans le tas), et nombre maximum de sommets dans le tas (vous pouvez considérer des caractéristiques supplémentaires).

\subsection{Cartes routières}

\subsection{Cartes non routières}

\section{Comparaison des performances entre Dijkstra et A$^*$}
En vous basant sur les résultats précédents comparez les caractéristiques des algorithmes en cout des solutions (distance et temps), temps de calcul, nombre de sommets explorés (ie entrant dans le tas), et nombre maximum de sommets dans le tas.
\\
Vous pouvez également faire varier le calcul de l'heuristique de A$^*$, séparer l'analyse selon des caractéristiques liées aux chemins, .....
\\
De part les algorithmes on s'attend à ce que l'algorithme $A^*$ soit plus "performant" que celui de Dijkstra. Est-ce toujours le cas dans votre analyse ? Justifiez vos réponses.

\subsection{Cartes routières}

\subsection{Cartes non routières}

\section{Conclusion}

\input{./annexes.tex}

\newpage

%récupérer les citation avec "/footnotemark"
\nocite{*}

%choix du style de la biblio
\bibliographystyle{plain}
%inclusion de la biblio
\bibliography{bibliographie}
%voir wiki pour plus d'information sur la syntaxe des entrées d'une bibliographie

\end{document}