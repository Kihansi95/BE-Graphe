\chapter{Validation des algorithmes de calcul de plus court chemin}

Introduction : expliquez (en 2 ou 3 phrases) l'objectif de ce chapitre et son organisation

\section{Tests sur le code produit}

\subsection{Contexte}
Précisez le contexte des tests effectués : 
\begin{itemize}
	\item cartes utilisées (routières, non routières, taille des cartes, ...),
	\item nombre de tests effectués sur chaque carte, 
	\item  déroulement des tests,
	\item ....
\end{itemize}

\subsection{Tests des cas d'erreur}
Traitement d'erreurs dans les données (sommets inexistants, arcs inexistants, arcs négatifs, ....)
\\
Expliquez comment vous avez procédé à ces vérifications.


\subsection{Absence de chemin ou chemins de cout nul}
Détection de l'absence de chemin entre deux sommets et de chemins de cout nul. 
\\
Expliquez comment vous avez procédé à ces vérifications.


\subsection{Cout des chemins (1)}
Le cout (en distance et en temps) retournés par les algorithmes (Dijkstra et A$^*$) est-il : 
\begin{itemize}
	\item identique ?
	\item le même que celui retourné par la fonction \texttt{calcul\_cout\_chemin} ?
\end{itemize}
Expliquez comment vous avez procédé à ces vérifications.
\newline

\subsection{Vérification du théorème sur les plus courts chemins}
Un sous-chemin d'un plus court chemin est-il un plus court chemin (en distance et en temps) ? 
\\
Expliquez comment vous avez procédé à ces vérifications.


\subsection{Cout des chemins (2) (option)}
\noindent
Comparez le cout en temps de plus courts chemins en distance par rapport aux plus courts chemins en temps.
\\
Comparez le cout en distance de plus courts chemins en distance par rapport aux plus courts chemins en distance.
\\
Expliquez comment vous avez procédé à ces vérifications.

\subsection{Inégalité triangulaire (option)}
L'inégalité triangulaire est-elle vérifiée ? 
\\
Expliquez comment vous avez procédé à ces vérifications.


\subsection{Bilan récapitulatif des tests}
Résumez les résultats obtenus.
\\
Expliquez comment reproduire vos tests à partir de votre code source

\section{Comparaisons croisées des résultats obtenus}
\subsection{Comparaisons avec d'autres binômes}
Résultats obtenus par rapport à d'autres binômes sur des couples (origine, destination) fixés : pour chaque algorithme vous devez donner le cout de vos solutions (dans les 2 cas, distance et temps), les temps de calcul, le nombre de sommets explorés (ie entrant dans le tas), et le nombre maximum de sommets dans le tas.

Dans le tableau~\ref{tab:validation-comparaison-resultats-distance}, donnez les résultats obtenus pour les plus courts chemins en distance et
dans le tableau~\ref{tab:validation-comparaison-resultats-temps}, donnez les résultats obtenus pour les plus courts chemins en temps.
%tableau centré à taille variable qui s'ajuste automatiquement suivant la longueur du contenu
\begin{figure}[!h]
\begin{center}
\begin{tabular}{|l||l|l|l|l||l|l|l|l|}
  \hline
    PCC Distance   & \multicolumn{4}{c||}{Algorithme Dijktra} & \multicolumn{4}{c|}{Algorithme A$^*$}\\
  \hline
	 Trajets & Cout & CPU & Nb Tas & Max Tas & Cout & CPU & Nb Tas & Max Tas \\
  \hline
	O1, D1 & & & & & & & & \\
	\hline
	O2, D2 & & & & & & & & \\
	\hline
	O3, D3 & & & & & & & & \\
	\hline
\end{tabular}
\end{center}
\caption{Tableau récapitulatif des solutions pour les PCC en distance}
\label{tab:validation-comparaison-resultats-distance}
\end{figure}

%tableau centré à taille variable qui s'ajuste automatiquement suivant la longueur du contenu
\begin{figure}[!h]
\begin{center}
\begin{tabular}{|l||l|l|l|l||l|l|l|l|}
  \hline
   PCC Temps & \multicolumn{4}{c||}{Algorithme Dijktra} & \multicolumn{4}{c|}{Algorithme A$^*$}\\
  \hline
	 Trajets & Cout & CPU & Nb Tas & Max Tas & Cout & CPU & Nb Tas & Max Tas \\
  \hline
	O1, D1 & & & & & & & & \\
	\hline
	O2, D2 & & & & & & & & \\
	\hline
	O3, D3 & & & & & & & & \\
	\hline
\end{tabular}
\end{center}
\caption{Tableau récapitulatif des solutions pour les PCC en temps}
\label{tab:validation-comparaison-resultats-temps}
\end{figure}

\paragraph*{Bilan de cette comparaison}

\subsection{Comparaisons avec un autre calculateur (option)}
Comparez vos résultats avec un calculateur d'itinéraires existant.

\subsection{Vérification visuelle (option)}

\subsection{Bilan récapitulatif du comparatif croisé}
Résumez les résultats obtenus.
\\
Expliquez comment reproduire vos tests à partir de votre code source

\section{Conclusion}
