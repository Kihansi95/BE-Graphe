\chapter{Analyse des performances}

Introduction : expliquez (en 2 ou 3 phrases) l'objectif de ce chapitre et son organisation

\section{Contexte expérimental}
Précisez le contexte des tests effectués : 
\begin{itemize}
	\item cartes utilisées (routières, non routières, taille des cartes, ...),
	\item nombre de tests effectués sur chaque carte, 
	\item  déroulement des tests,
	\item ....
\end{itemize}
Expliquez comment reproduire vos tests de performance à partir de votre code source

\section{Caractéristiques de l'algorithme Dijkstra}
Evaluez les caractéristiques de l'algorithme en cout des solutions (distance et temps), temps de calcul, nombre de sommets explorés (ie entrant dans le tas), et nombre maximum de sommets dans le tas (vous pouvez considérer des caractéristiques supplémentaires).

\subsection{Cartes routières}

\subsection{Cartes non routières}

\section{Caractéristiques de l'algorithme A$^*$}
Evaluez les caractéristiques de l'algorithme en cout des solutions (distance et temps), temps de calcul, nombre de sommets explorés (ie entrant dans le tas), et nombre maximum de sommets dans le tas (vous pouvez considérer des caractéristiques supplémentaires).

\subsection{Cartes routières}

\subsection{Cartes non routières}

\section{Comparaison des performances entre Dijkstra et A$^*$}
En vous basant sur les résultats précédents comparez les caractéristiques des algorithmes en cout des solutions (distance et temps), temps de calcul, nombre de sommets explorés (ie entrant dans le tas), et nombre maximum de sommets dans le tas.
\\
Vous pouvez également faire varier le calcul de l'heuristique de A$^*$, séparer l'analyse selon des caractéristiques liées aux chemins, .....
\\
De part les algorithmes on s'attend à ce que l'algorithme $A^*$ soit plus "performant" que celui de Dijkstra. Est-ce toujours le cas dans votre analyse ? Justifiez vos réponses.

\subsection{Cartes routières}

\subsection{Cartes non routières}

\section{Conclusion}
